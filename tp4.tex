\documentclass[a4paper,12pt]{report}
\usepackage[utf8]{inputenc}
\usepackage{enumitem} %permite el uso de letras para enumerar
\usepackage{graphicx} %para las imágenes
\usepackage{float} %para fijar las imágenes

\usepackage{tikz}
\usetikzlibrary{arrows.meta, positioning} %para hacer diagramas de bloques

\usepackage{amsmath}%para entornos de alineación
\usepackage{amsfonts}%para las letras lindas de matemática
\setlength{\jot}{8pt}%modifica el interlineado

\usepackage{tikz} %Librería para gráficos
\usetikzlibrary{calc, arrows.meta, positioning}

\usepackage[a4paper, %margenes de pagina
  left=2.5cm,
  right=2.5cm,
  top=2cm,
  bottom=2cm,
  includehead
]{geometry}

\usepackage{fancyhdr}
\pagestyle{fancy}
\lhead{UTN-FRC}
\chead{ASyS}
\rhead{2R3}
\cfoot{\thepage}
\setlength{\headwidth}{\textwidth} % Hace que el ancho del encabezado coincida con el ancho del texto
\setlength{\headheight}{15pt}  % Ajusta la altura del encabezado
\setlength{\headsep}{20pt}     % Ajusta la separación entre el encabezado y el contenido

\usepackage{titlesec}
\titleformat{\chapter}[display]
  {\normalfont\Large\bfseries}{}{0pt}{}
\titlespacing*{\chapter}{10pt}{-45pt}{10pt}

\usepackage{etoolbox} 
\makeatletter
\patchcmd{\chapter}{\thispagestyle{plain}}{\thispagestyle{fancy}}{}{} %Muestra encabezado en las paginas con \chapter
\makeatother

%Comandos de fake section y fake sub section, para poder agregar secciones al indice
\newcommand{\fs}[1]{%
  \par\refstepcounter{section}% Increase section counter
  \sectionmark{#1}% Add section mark (header)
  \addcontentsline{toc}{section}{\protect\numberline{\thechapter.\alph{section}}#1}% Add section to ToC
}
\newcommand{\fss}[1]{%
  \par\refstepcounter{subsection}% Increase subsection counter
  \subsectionmark{#1}% Add subsection mark (header)
  \addcontentsline{toc}{subsection}{\protect\numberline{\alph{subsection}}#1}% Add subsection to ToC
}

\renewcommand{\contentsname}{Tabla de Contenidos}

\usepackage{afterpage}
\newcommand\myemptypage{
  \newpage
  \null
  \thispagestyle{empty}
  \addtocounter{page}{-1}
  \newpage
}

\title{%
\setlength{\headwidth}{\textwidth} % Hace que el encabezado tenga el mismo ancho que el contenido
\setlength{\headheight}{15pt}  % Ajusta la altura del encabezado
\setlength{\headsep}{10pt}     % Ajusta la separación entre el encabezado y el contenido
  \fontsize{25}{0}\selectfont Universidad Tecnológica Nacional \\
  \fontsize{22}{30}\selectfont Análisis de Señales y Sistemas \\
  \fontsize{20}{25}\selectfont Trabajo Practico 4
}
\author{
    Franco Palombo - 401910\\
    Ignacio Gil - 401891\\
    Laureano Valentin Reinoso - 402075\\
    Luciano Tomas Cortesini Perez - 402719\\
}
\date{18 / 11 / 2024}

\begin{document}

    \maketitle

    \myemptypage

    \tableofcontents
    \thispagestyle{plain}

    \myemptypage

    \chapter{Ejercicio 1}
        Considerar la ecuacion diferencial del TP2, Ejercicio 4 (Circuito Electrico RLC) tratadas por medio de
        convolucion temporal y Transformada de Laplace:

        \begin{enumerate}[label=\alph*), left=0pt]
            \item \fs{} ¿Que ancho de banda tiene el filtro en tiempo continuo?

            \item \fs{} ¿Que frecuencia de muestreo deberia aplicarse para no perder informacion muestreando este
                sistema? Evaluar hasta caida de 3dB como banda de paso.

            \item \fs{} Considerar $T_1 = 1s$ y $T_2 = 0,1s$. Proceder a la discretizacion por dos metodos diferentes,
                incluida la transformacion bilineal.

            \item \fs{} Obtener la $H_{(z)}$.

            \item \fs{} Desarrollar la respuesta $h_{[n]}$ al impulso $\delta_{[n]}$ por medio de la transformada Z.

            \item \fs{} Desarrollar la respuesta $\mu_{[n]}$ al impulso $y_{[n]}$ por medio de la transformada Z.

            \item \fs{} Verificar el TVI y TVF para ambas respuestas en ambos dominios.

            \item \fs{} Comparar graficamente las respuestas en tiempo continuo y tiempo discreto.

            \item \fs{} Comparar los TVI y TVF para ambas respuestas, continua y discreta.

            \item \fs{} Bosquejar el diagrama de BODE correspondiente para los dos filtros, y determimar $\omega_c$
                del filtro.

            \item \fs{} De acuerdo a los resultados obtenidos, ¿Que tipo de filtro es?

        \end{enumerate}

\end{document}
