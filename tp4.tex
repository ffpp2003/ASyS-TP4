\documentclass[a4paper,12pt]{report}
\usepackage[utf8]{inputenc}
\usepackage{enumitem} %permite el uso de letras para enumerar
\usepackage{graphicx} %para las imágenes
\usepackage{float} %para fijar las imágenes

\usepackage{tikz}
\usetikzlibrary{arrows.meta, positioning} %para hacer diagramas de bloques

\usepackage{amsmath}%para entornos de alineación
\usepackage{amsfonts}%para las letras lindas de matemática
\setlength{\jot}{8pt}%modifica el interlineado

\usepackage{tikz} %Librería para gráficos
\usetikzlibrary{calc, arrows.meta, positioning}

\usepackage[a4paper, %margenes de pagina
  left=2.5cm,
  right=2.5cm,
  top=2cm,
  bottom=2cm,
  includehead
]{geometry}

\usepackage{fancyhdr}
\pagestyle{fancy}
\lhead{UTN-FRC}
\chead{ASyS}
\rhead{2R3}
\cfoot{\thepage}
\setlength{\headwidth}{\textwidth} % Hace que el ancho del encabezado coincida con el ancho del texto
\setlength{\headheight}{15pt}  % Ajusta la altura del encabezado
\setlength{\headsep}{20pt}     % Ajusta la separación entre el encabezado y el contenido

\usepackage{titlesec}
\titleformat{\chapter}[display]
  {\normalfont\Large\bfseries}{}{0pt}{}
\titlespacing*{\chapter}{10pt}{-45pt}{10pt}

\usepackage{etoolbox} 
\makeatletter
\patchcmd{\chapter}{\thispagestyle{plain}}{\thispagestyle{fancy}}{}{} %Muestra encabezado en las paginas con \chapter
\makeatother

%Comandos de fake section y fake sub section, para poder agregar secciones al indice
\newcommand{\fs}[1]{%
  \par\refstepcounter{section}% Increase section counter
  \sectionmark{#1}% Add section mark (header)
  \addcontentsline{toc}{section}{\protect\numberline{\thechapter.\alph{section}}#1}% Add section to ToC
}
\newcommand{\fss}[1]{%
  \par\refstepcounter{subsection}% Increase subsection counter
  \subsectionmark{#1}% Add subsection mark (header)
  \addcontentsline{toc}{subsection}{\protect\numberline{\alph{subsection}}#1}% Add subsection to ToC
}

\renewcommand{\contentsname}{Tabla de Contenidos}

\usepackage{afterpage}
\newcommand\myemptypage{
  \newpage
  \null
  \thispagestyle{empty}
  \addtocounter{page}{-1}
  \newpage
}

\usepackage{circuitikz}

\title{%
\setlength{\headwidth}{\textwidth} % Hace que el encabezado tenga el mismo ancho que el contenido
\setlength{\headheight}{15pt}  % Ajusta la altura del encabezado
\setlength{\headsep}{10pt}     % Ajusta la separación entre el encabezado y el contenido
  \fontsize{25}{0}\selectfont Universidad Tecnológica Nacional \\
  \fontsize{22}{30}\selectfont Análisis de Señales y Sistemas \\
  \fontsize{20}{25}\selectfont Trabajo Practico 4
}
\author{
    Franco Palombo - 401910\\
    Ignacio Gil - 401891\\
    Laureano Valentin Reinoso - 402075\\
    Luciano Tomas Cortesini Perez - 402719\\
}
\date{18 / 11 / 2024}

\begin{document}

    \maketitle

    \myemptypage

    \tableofcontents
    \thispagestyle{plain}

    \myemptypage

    \chapter{Ejercicio 1}
        Considerar la ecuacion diferencial del TP2, Ejercicio 4 (Circuito Electrico RLC) tratadas por medio de
        convolucion temporal y Transformada de Laplace:

        \begin{enumerate}[label=\alph*), left=0pt]
            \item \fs{} ¿Que ancho de banda tiene el filtro en tiempo continuo?\\

                \begin{figure}[h]
                    \centering
                    \begin{circuitikz}[american voltages]
                        % nodos
                        \draw
                            (0, 0)  to [short, o-, on grid]                 (1, 0)
                                    to [R, l_=$R$, on grid]                 (3, 0)
                                    to [L, l_=$L$, on grid]                 (7, 0)
                                    to [C, l_=$C$, on grid]                 (7, -3)
                                    to [short, on grid]                     (7, -3)
                                    to [short, on grid]                     (0, -3)
                                    to [short, o-, on grid]                 (0, -3)
                            (0, 0)  to [open, v^>=$v_{(t)}$, on grid]       (0, -3)
                            (7, 0)  to [short, *-, on grid]                 (9, 0)
                                    to [short, o-, on grid]                 (9, 0)
                            (7, -3) to [short, *-, on grid]                 (9, -3)
                                    to [short, o-, on grid]                 (9, -3)
                            (9, 0)  to [open, v^>=$v_{c(t)}$, on grid]      (9, -3)
                            ;
                    \end{circuitikz}

                    \textit{Diagrama del circuito RLC.}
                \end{figure}

                Para determinar el ancho de banda del circuito, podemos utilizar la siguiente inecuacion:
                \begin{equation}
                    \label{ancho.de.banda}
                    \lvert H_{(\omega)} \rvert > \frac{1}{\sqrt{2}}
                \end{equation}

                Para determinar la funcion de transferencia del sistema, primero debemos modelar el circuito. Sabiendo
                que la tension $v(t)$ esta determinada por:
                \begin{equation*}
                    v_{(t)} = Ri_{(t)} + L \frac{d}{dt} i_{(t)} + \frac{1}{C} \int_{0}^{t} i_{(\tau)} d\tau
                \end{equation*}

                Si aplicamos la transformada de Fourier tal cual esta la ecuacion, nos quedaria:
                \begin{equation*}
                    \mathcal{F} \left\{ v_{(t)} \right\} = V_{(\omega)} = R I_{(\omega)} + j \omega L I_{(\omega)} +
                        \frac{1}{j \omega C} I_{(\omega)}
                \end{equation*}

                Se sabe ademas, que la impedancia de un elemento electrico esta definida por:
                \begin{equation}
                    \label{impedancia}
                    Z_{(\omega)} = \frac{V_{(\omega)}}{I_{(\omega)}}
                \end{equation}

                Por lo tanto, si en $V_{(\omega)}$ multiplicamos de ambos lados por $\frac{1}{I_{(\omega)}}$:
                \begin{equation*}
                    Z_{T(\omega)} = R + j \omega L + \frac{1}{j \omega C}\\
                \end{equation*}

                Por lo que podemos decir que:
                \begin{align*}
                    Z_{R(\omega)} = R \hspace{1cm}\quad
                    Z_{L(\omega)} = j \omega L \hspace{1cm}\quad
                    Z_{C(\omega)} = \frac{1}{j \omega C}
                \end{align*}

                Acomodando $Z_{T(\omega)}$:
                \begin{equation*}
                    Z_{T(\omega)} = \frac{1 + j \omega R C - \omega^2 L C}{j \omega C}
                \end{equation*}

                Para determinar $v_{c(t)}$, necesitamos, en el dominio de la frecuencia, saber la caida de potencial
                que se genera en la resistencia y en el inductor. Para ello, debemos obtener la funcion de corriente
                $I_{(\omega)}$ para poder obtener las caidas de tension de la resistencia y el inductor utilizando la
                ley de Ohm. La corriente esta definida por:
                \begin{equation}
                    \label{corriente}
                    I_{(\omega)} = \frac{V_{(\omega)}}{Z_{(\omega)}}
                \end{equation}

                Entonces:
                \begin{equation*}
                    I_{(\omega)} = V_{(\omega)} \frac{j \omega C}{1 + j \omega R C - \omega^2 L C}
                \end{equation*}

                Podemos entonces determinar las caidas de tension individuales segun las leyes de Kirchoff:
                \begin{figure}[h]
                    \centering
                    \begin{minipage}{0.4\textwidth}
                        \begin{align*}
                            V_{R(\omega)} &= I_{(\omega)} Z_{R(\omega)}\\
                            V_{R(\omega)} &= R V_{(\omega)} \frac{j \omega C}{1 + j \omega R C - \omega^2 L C}\\
                            V_{R(\omega)} &= V_{(\omega)} \frac{j \omega R C}{1 + j \omega R C - \omega^2 L C}
                        \end{align*}
                    \end{minipage}
                    \hspace{1cm}
                    \begin{minipage}{0.4\textwidth}
                        \begin{align*}
                            V_{L(\omega)} &= I_{(\omega)} Z_{L(\omega)}\\
                            V_{L(\omega)} &= j \omega L V_{(\omega)} \frac{j \omega C}{1 + j \omega R C - \omega^2 L C}\\
                            V_{L(\omega)} &= V_{(\omega)} \frac{-\omega^2 L C}{1 + j \omega R C - \omega^2 L C}
                        \end{align*}
                    \end{minipage}
                \end{figure}

                Por Kirchoff:
                \begin{align*}
                    V_{C(\omega)} &= V_{(\omega)} - \left( V_{R(\omega)} + V_{L(\omega)} \right)\\
                    V_{C(\omega)} &= V_{(\omega)} - \left( V_{(\omega)} \frac{j \omega R C}{1 + j \omega R C -
                        \omega^2 L C} + V_{(\omega)} \frac{-\omega^2 L C}{1 + j \omega R C - \omega^2 L C} \right)\\
                    V_{C(\omega)} &= V_{(\omega)} \left( 1 - \frac{j \omega R C - \omega^2 L C}{1 + j \omega R C -
                        \omega^2 L C} \right)\\
                    \frac{V_{C(\omega)}}{V_{(\omega)}} &= H_{(\omega)} = \frac{1}{1 + j \omega R C - \omega^2 L C}\\
                \end{align*}

            \item \fs{} ¿Que frecuencia de muestreo deberia aplicarse para no perder informacion muestreando este
                sistema? Evaluar hasta caida de 3dB como banda de paso.\\

            \item \fs{} Considerar $T_1 = 1s$ y $T_2 = 0,1s$. Proceder a la discretizacion por dos metodos diferentes,
                incluida la transformacion bilineal.\\

            \item \fs{} Obtener la $H_{(z)}$.\\

            \item \fs{} Desarrollar la respuesta $h_{[n]}$ al impulso $\delta_{[n]}$ por medio de la transformada Z.\\

            \item \fs{} Desarrollar la respuesta $\mu_{[n]}$ al impulso $y_{[n]}$ por medio de la transformada Z.\\

            \item \fs{} Verificar el TVI y TVF para ambas respuestas en ambos dominios.\\

            \item \fs{} Comparar graficamente las respuestas en tiempo continuo y tiempo discreto.\\

            \item \fs{} Comparar los TVI y TVF para ambas respuestas, continua y discreta.\\

            \item \fs{} Bosquejar el diagrama de BODE correspondiente para los dos filtros, y determimar $\omega_c$
                del filtro.\\

            \item \fs{} De acuerdo a los resultados obtenidos, ¿Que tipo de filtro es?\\

        \end{enumerate}

\end{document}
